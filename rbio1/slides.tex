% Copyright (c) 2022 by Lars Spreng
% This work is licensed under the Creative Commons Attribution 4.0 International License. 
% To view a copy of this license, visit http://creativecommons.org/licenses/by/4.0/ or send a letter to Creative Commons, PO Box 1866, Mountain View, CA 94042, USA.

%~~~~~~~~~~~~~~~~~~~~~~~~~~~~~~~~~~~~~~~~~~~~~~~~~~~~~~~~~~~~~~~~~~~~~~~~~~~~~~
% You can add your packages and commands to the loadslides.tex file. 
% The files in the folder "styles" can be modified to change the layout and design of your slides.
% I have included examples on how to use the template below. 
% Some of these examples are taken from the Metropolis template.
%~~~~~~~~~~~~~~~~~~~~~~~~~~~~~~~~~~~~~~~~~~~~~~~~~~~~~~~~~~~~~~~~~~~~~~~~~~~~~~


\documentclass[
11pt,notheorems,hyperref={pdfauthor=whatever}
]{beamer}


% Copyright (c) 2022 by Lars Spreng
% This work is licensed under the Creative Commons Attribution 4.0 International License.
% To view a copy of this license, visit http://creativecommons.org/licenses/by/4.0/ or send a letter to Creative Commons, PO Box 1866, Mountain View, CA 94042, USA.

%~~~~~~~~~~~~~~~~~~~~~~~~~~~~~~~~~~~~~~~~~~~~~~~~~~~~~~~~~~~~~~~~~~~~~~~~~~~~~~
% Add your packages and commands to this file
%~~~~~~~~~~~~~~~~~~~~~~~~~~~~~~~~~~~~~~~~~~~~~~~~~~~~~~~~~~~~~~~~~~~~~~~~~~~~~~

%~~~~~~~~~~~~~~~~~~~~~~~~~~~~~~~~~~~~~~~~~~~~~~~~~~~~~~~~~~~~~~~~~~~~~~~~~~~~~~
% Fonts
\RequirePackage[scaled]{helvet} % for sans-serif slides

\RequirePackage[utf8]{inputenc}
\RequirePackage[T1]{fontenc}


\usepackage{styles/elegantmacros}
\usefolder{styles}
\usetheme[style=lecture]{elegant}

\newcommand{\makepart}[1]{ % For convenience
\part{#1} \frame{\partpage}
}

%~~~~~~~~~~~~~~~~~~~~~~~~~~~~~~~~~~~~~~~~~~~~~~~~~~~~~~~~~~~~~~~~~~~~~~~~~~~~~~

%~~~~~~~~~~~~~~~~~~~~~~~~~~~~~~~~~~~~~~~~~~~~~~~~~~~~~~~~~~~~~~~~~~~~~~~~~~~~~~
% Figures
\RequirePackage{booktabs}
\RequirePackage{colortbl}
\RequirePackage{ragged2e}
\RequirePackage{caption}
\RequirePackage{subcaption}
\RequirePackage{tabularx}
\RequirePackage{array}
\RequirePackage{multirow}

\newcolumntype{Y}{>{\centering\arraybackslash}X}

%~~~~~~~~~~~~~~~~~~~~~~~~~~~~~~~~~~~~~~~~~~~~~~~~~~~~~~~~~~~~~~~~~~~~~~~~~~~~~~

%~~~~~~~~~~~~~~~~~~~~~~~~~~~~~~~~~~~~~~~~~~~~~~~~~~~~~~~~~~~~~~~~~~~~~~~~~~~~~~
% Figures
\RequirePackage{wrapfig}
\RequirePackage{graphicx}
\RequirePackage{adjustbox}

%~~~~~~~~~~~~~~~~~~~~~~~~~~~~~~~~~~~~~~~~~~~~~~~~~~~~~~~~~~~~~~~~~~~~~~~~~~~~~~

%~~~~~~~~~~~~~~~~~~~~~~~~~~~~~~~~~~~~~~~~~~~~~~~~~~~~~~~~~~~~~~~~~~~~~~~~~~~~~~
% Code listings
\RequirePackage{listings}
\RequirePackage{textcomp}

\lstdefinestyle{terminalstyle}{
  basicstyle=\ttfamily\scriptsize,
  backgroundcolor=\color{black!5},
  frame=single,
  rulecolor=\color{black!30},
  framesep=5pt,
  breaklines=true,
  breakatwhitespace=false,
  showstringspaces=false,
  columns=fullflexible,
  keepspaces=true,
  xleftmargin=5pt,
  xrightmargin=5pt,
  aboveskip=8pt,
  belowskip=4pt,
  keywordstyle=\color{primary}\bfseries,
  commentstyle=\color{Gray}\itshape,
}

\lstdefinestyle{yamlstyle}{
  basicstyle=\ttfamily\scriptsize,
  backgroundcolor=\color{black!3},
  frame=single,
  rulecolor=\color{primary!30},
  framesep=5pt,
  breaklines=true,
  showstringspaces=false,
  columns=fullflexible,
  keepspaces=true,
  xleftmargin=5pt,
  xrightmargin=5pt,
  aboveskip=8pt,
  belowskip=4pt,
  keywordstyle=\color{primary}\bfseries,
  stringstyle=\color{secondary},
  commentstyle=\color{Gray}\itshape,
  morecomment=[l]{\#},
}

%~~~~~~~~~~~~~~~~~~~~~~~~~~~~~~~~~~~~~~~~~~~~~~~~~~~~~~~~~~~~~~~~~~~~~~~~~~~~~~

%~~~~~~~~~~~~~~~~~~~~~~~~~~~~~~~~~~~~~~~~~~~~~~~~~~~~~~~~~~~~~~~~~~~~~~~~~~~~~~
% TikZ for diagrams
\RequirePackage{tikz}
\usetikzlibrary{arrows.meta, positioning, shapes.geometric, calc, fit}

%~~~~~~~~~~~~~~~~~~~~~~~~~~~~~~~~~~~~~~~~~~~~~~~~~~~~~~~~~~~~~~~~~~~~~~~~~~~~~~
 % Loads packages and some defined commands

\usepackage{tabularx}
\usepackage{array}
\usepackage{booktabs}


\title[
% Text entered here will appear in the bottom middle
]{rbio1 - training scientific reasoning LLMs with biological world models as soft verifiers}


\author[
% Text entered here will appear in the bottom left corner
]{
    Istrate \textit{et al.}
}

\institute{Chan Zuckerberg Initative}
\date{August 22, 2025}

\begin{document}

% Generate title page
{
\setbeamertemplate{footline}{} 
\begin{frame}
  \titlepage
\end{frame}
}
\addtocounter{framenumber}{-1}

% You can declare different parts as a parentof sections

\section{Abstract}

\begin{frame}
Reasoning Models are typically trained against verification 
mechanisms in formally specified systems such as code or symbolic 
math. However, in open domains like biology, we do not generally 
have access to exact rules facilitating formal verification at scale, 
and oftentimes resolve to testing hypotheses in the lab to assess the 
validity of a prediction. 

Verification by performing real experiments is slow, expensive, and 
inherently does not scale with computation.

In this work, we show that one can use \textbf{world models of biology or 
other prior knowledge as approximate oracles over biological 
knowledge to utilize as soft verification to train reasoning systems 
without the need for additional experimental data.}

We introduce rbio1, \textbf{a reasoning model for biology that is post-trained 
from a pretrained LLM using reinforcement learning and uses learned 
models of biology to obtain biological knowledge for verification 
during training.} We show that soft verification successfully distills 
biology world models into rbio, at the example of achieving leading 
performance on perturbation prediction against the PerturbQA 
benchmark compared to state-of-the-art models; we demonstrate the 
benefits of compositions of verifiers to learn more general rbio 
models. 

We believe rbio provides a proof of concept that demonstrates that 
predictions from bio-models can be used to train powerful reasoning 
models using simulations, rather than experimental data, as a new 
training paradigm.
\end{frame}

\section{Introduction}
\begin{frame}{Introduction: Motivation and Core Idea}

\begin{itemize}
  \item \textbf{Problem:} Experimental biological supervision does not scale
  \begin{itemize}
    \item Experiments are expensive, slow, and sparse
    \item Many biologically meaningful questions lack labels
  \end{itemize}

  \vspace{0.3em}

  \item \textbf{Limitation of current LLM training}
  \begin{itemize}
    \item Supervised fine-tuning assumes access to ground truth
    \item Fails when labels are unavailable or incomplete
  \end{itemize}

  \vspace{0.3em}

  \item \textbf{Key insight}
  \begin{itemize}
    \item Biology already exists in multiple forms:
    \begin{itemize}
      \item Experiments
      \item Predictive models (simulations)
      \item Curated knowledge
    \end{itemize}
    \item These can act as \textbf{verifiers}, not labels
  \end{itemize}

  \vspace{0.3em}

  \item \textbf{Core idea of the paper}
  \begin{itemize}
    \item Replace ground truth with \textbf{verification signals}
    \item Learn biological reasoning via \textbf{reinforcement learning}
  \end{itemize}

\end{itemize}

\end{frame}

\section{The overal structure of rbio1}
\begin{frame}
    \begin{figure}[htbp]
        \centering
        \includegraphics[width=1.0\textwidth]{figures/rbio1-overview.png}
    \end{figure}
\end{frame}

\begin{frame}
    \begin{figure}[htbp]
        \centering
        \includegraphics[width=0.9\textwidth]{figures/rbio1-overview-verification.png}
    \end{figure}

    \begin{itemize}
        \item \texttt{input a} and \texttt{input b}?
        \item What are the hard/soft verifiers?
        \item How is the reward computed?
    \end{itemize}
\end{frame}

\begin{frame}{Table 1: the key to understand the rest of paper}
    \begin{figure}[htbp]
        \centering
        \includegraphics[width=0.8\textwidth]{figures/rbio1-table1.png}
    \end{figure}
\end{frame}


\begin{frame}{How the training works}
    \begin{figure}[htbp]
        \centering
        \includegraphics[width=0.5\textwidth]{figures/rbio1-overview-training.png}
    \end{figure}
\end{frame}

\section{Verifiers for rbio1}
\begin{frame}
\begin{table}[ht]
\footnotesize
\renewcommand{\arraystretch}{0.9}

\centering
\begin{tabular}{|c|l|l|l|l|l|c|l|}
\hline
 \textbf{Verifier source} & \textbf{Output form} & \textbf{Compared against} & \textbf{Reward meaning} & \textbf{Strength of truth} & \textbf{Why it counts as truth} \\
\hline
Experimental data  & Binary label & LLM yes/no & Correctness & $\star \star \star \star \star$  & Direct observation \\
 Virtual Cell Model  & Probability & LLM yes/no & Consistency & $\star \star \star \star$  & Learned biological structure \\
Knowledge Source  & Text & LLM text & Alignment & $\star \star \star$  & Expert consensus \\
\hline
\end{tabular}
\caption{Truth verification signals used in Sections 2.2--2.4}
\end{table}

\end{frame}

\section{Equations for rbio1}
\begin{frame} 

\scriptsize
\setlength{\tabcolsep}{3pt}
\renewcommand{\arraystretch}{1.1}

\begin{tabularx}{\textwidth}{|
  >{\raggedright\arraybackslash}p{3.0cm}|
  >{\raggedright\arraybackslash}X|
  >{\raggedright\arraybackslash}X|
  >{\raggedright\arraybackslash}X|
  >{\raggedright\arraybackslash}p{3.3cm}|
}
\hline
\textbf{Mechanism} &
\textbf{Purpose} &
\textbf{What it stabilizes} &
\textbf{Failure mode without it} &
\textbf{One idea / equation} \\
\hline

GRPO (2.1) &
Learn from rewards &
Policy updates &
No learning &
$\hat{A}_i=\dfrac{r_i-\mu}{\sigma}$ \\

\hline
Composable rewards (2.5) &
Combine signals &
Generalization &
Overfitting to one truth &
$r=\sum_j r_j$ \\

\hline
Score normalization (2.6) &
Align reward scales &
Optimization stability &
Reward collapse &
Piecewise min--max $\rightarrow [0,1]$ \\
\hline
\end{tabularx}


\end{frame}

\begin{frame}{Metric definitions}
  \begin{figure}[htbp]
    \centering
    \includegraphics[width=0.85\textwidth]{figures/rbio1-metric.png}
  \end{figure}
\end{frame}

\begin{frame}{Experimental setup}
  \begin{figure}[htbp]
    \centering
    \begin{minipage}{0.3\textwidth}
      \centering
      \includegraphics[width=\textwidth]{figures/rbio1-training-setup.png}
      \vspace{0.4em}
    \end{minipage}\hfill
    \begin{minipage}{0.7\textwidth}
      \centering
      \includegraphics[width=\textwidth]{figures/rbio1-all-cell-lines.png}
      \vspace{0.4em}
    \end{minipage}
  \end{figure}
\end{frame}

\section{Results summary}

\begin{frame}{Paper Roadmap by Figure}

\scriptsize
\setlength{\tabcolsep}{4pt}
\renewcommand{\arraystretch}{1.15}

\begin{tabularx}{\textwidth}{
  l
  c
  >{\raggedright\arraybackslash}X
  >{\raggedright\arraybackslash}X
  >{\raggedright\arraybackslash}X
  >{\raggedright\arraybackslash}X
}
\toprule
\textbf{Figure} &
\textbf{Sec.} &
\textbf{Core question} &
\textbf{Training detail (specific point)} &
\textbf{Purpose} &
\textbf{Conclusion} \\
\midrule

Fig.~2 & 3.3 &
Can soft verification (MLP) match training on experimental labels on-task? &
Train rbio using \textbf{MLP soft scores (VCM)} instead of EXP labels; leave-one-out cell line logic &
Validate \textbf{VCM-as-verifier} for perturbation prediction &
Soft-verifier training can be competitive with EXP training (esp.\ recall/TPR tradeoffs) \\

Fig.~3 & 3.4 &
Can off-task VCM biology transfer to perturbation prediction? &
Train on \textbf{Transcriptformer co-expression / PMI} prompts; test on perturbation &
Show \textbf{transfer} from general biology signals &
Off-task biological tutoring improves downstream perturbation behavior \\

Fig.~4 & 3.5 &
Do multiple verifiers improve results vs.\ single verifiers? &
Train with \textbf{combinations of TF + GO + MLP} (optionally ESM); prompt mix skewed by dataset density &
Demonstrate \textbf{composable verification} at scale &
Adding diverse verification sources improves generalization/performance \\

Fig.~5 & 3.5 &
What happens in \textbf{ablation-style additions}? Any surprises? &
Stepwise addition of verifiers; some cases where GO hurts in certain combos &
Deepen \textbf{composition analysis} beyond Fig.~4 &
Generally additive gains, with some counterintuitive interactions flagged \\

Fig.~6 & 3.6 &
Does chain-of-thought prompting at inference boost test performance? &
\textbf{No retraining}; only test-time prompting style changes &
Show reasoning helps at inference &
CoT improves performance without tools or extra data \\

\bottomrule
\end{tabularx}

\end{frame}

\section{Conclusions}

\begin{frame}{Conclusion: What This Paper Shows}

\begin{itemize}
  \item \textbf{Biological reasoning can be learned without experiments}
  \begin{itemize}
    \item Soft verification from models and knowledge is sufficient
    \item Experimental data is no longer the only training signal
  \end{itemize}

  \vspace{0.3em}

  \item \textbf{Supervision need not be binary ground truth}
  \begin{itemize}
    \item Experiments provide hard labels
    \item Models and knowledge provide soft, graded signals
    \item All can guide learning via verification
  \end{itemize}

  \vspace{0.3em}

  \item \textbf{Compositional verification works}
  \begin{itemize}
    \item Multiple weak signals outperform any single source
    \item Biological supervision is additive, not exclusive
  \end{itemize}

  \vspace{0.3em}

  \item \textbf{Reasoning matters at training and inference}
  \begin{itemize}
    \item Reinforcement learning aligns reasoning with biology
    \item Chain-of-thought improves test-time performance
  \end{itemize}

  
\end{itemize}

\end{frame}

\section{Rebuttal Overview}

\begin{frame}{Rebuttal Overview: Main Criticism Themes}

\begin{itemize}
  \item \textbf{Verifier fidelity \& noise:} are soft verifiers reliable or do they amplify errors?
  \item \textbf{Generality:} is evidence limited to PerturbQA and closely-coupled rewards?
  \item \textbf{RL vs supervised baseline:} is GRPO actually needed beyond imitation?
  \item \textbf{Clarity \& reproducibility:} missing MLP details, unclear equations, missing compute.
  \item \textbf{Metrics/statistical claims:} what is ``SOTA'' under class imbalance and data fraction?
\end{itemize}

\vspace{0.4em}
\textbf{Author strategy (from official rebuttal):}
\begin{itemize}
  \item ``\emph{new controlled-noise, ablation, composition, and cross-domain experiments}'' 
  \item ``\emph{detailed training, scaling, and reproducibility analyses}'' (all in SI)
\end{itemize}

\end{frame}


\begin{frame}{Reviewer cju1: Verifier Fidelity \& Miscalibration}

\textbf{Reviewer cju1 (Weaknesses, verbatim):}
\begin{quote}
The approach assumes biology models and GO signals are accurate enough to guide RL, 
but the paper does not analyze sensitivity to verifier fidelity or miscalibration...
\end{quote}
\begin{quote}
There is no analysis of whether RL amplifies verifier errors, no plot of verifier vs model accuracy on held-out perturbations, 
and no check of verifier disagreement.
\end{quote}

\vspace{0.4em}
\textbf{Authors (Response, verbatim highlights):}
\begin{quote}
...controlled-noise experiments ... where MLP verifier predictions were progressively randomized... 
rbio performance decreases smoothly and remains well above the Qwen 2.5-3B baseline until signals are fully random...
\end{quote}
\begin{quote}
...confidence--performance analysis ... confirms that recall increases with verifier confidence...
indicating that rbio learns a stable biological prior rather than overfitting to verifier noise.
\end{quote}

\vspace{0.4em}
\textbf{What they added (SI):} controlled-noise (A.4.1), reward ablations (A.4.2), confidence analysis (A.4.3), cross-verifier agreement (A.5).

\vspace{0.2em}
\textbf{Takeaway for students:} When criticized for ``oracle quality'', add \emph{controlled-noise} + \emph{error amplification checks}.

\end{frame}

\begin{frame}{Reviewer cju1: Are Gains Just Generic RL?}

\textbf{Reviewer cju1 (Weakness, verbatim):}
\begin{quote}
It is unclear whether generic RL or RLAIF signals (format, helpfulness, self-consistency) would yield similar gains.
This weakens the claim that the biology-specific rewards are the true driver of improvement.
\end{quote}

\vspace{0.4em}
\textbf{Authors (Response, verbatim highlights):}
\begin{quote}
Reward ablations ... show that biological-answer rewards dominate performance gains, while format- or mention-only terms have minimal effect...
\end{quote}
\begin{quote}
...we compare ... DeepSeek R1, Qwen Instruct, and OpenAI OSS ... these models fail to reproduce the gains achieved by biology-grounded rewards...
\end{quote}

\vspace{0.4em}
\textbf{Takeaway for students:} If reviewers suspect ``it’s just RL regularization'', do:
\begin{itemize}
  \item ablation: domain reward vs format/mention rewards
  \item stronger external baselines: instruction-tuned / RL reasoning models
\end{itemize}

\end{frame}

\begin{frame}{Reviewer jVCL: Clarity, Biological Setup, and Ill-Defined Equations}

\textbf{Reviewer jVCL (Weaknesses, verbatim):}
\begin{quote}
The exposition needs to be improved strongly: the biological question and setup is not explained...
\end{quote}
\begin{quote}
Key details are not included in the paper: how was the tiny MLP trained (input / output / loss?)...
\end{quote}
\begin{quote}
Equations are often nonsensical / unhelpful / ill-defined... The paper cannot be reimplemented...
\end{quote}

\vspace{0.4em}
\textbf{Authors (Response, verbatim highlights):}
\begin{quote}
We now explicitly describe the PerturbQA benchmark in A.1...
\end{quote}
\begin{quote}
We added full architecture and hyperparameters ... and ... describe how the MLP’s probabilistic outputs serve as rewards during GRPO optimization.
\end{quote}
\begin{quote}
We ... re-worked the Methods section for clarity---simplifying notation, improving equation definitions...
\end{quote}

\vspace{0.4em}
\textbf{Takeaway for students:} Clarity rebuttals must add \emph{reimplementation-critical details}: dataset protocol, model IO/loss, algorithms, and simplified notation.

\end{frame}

\begin{frame}{Reviewer jVCL: Generalization Beyond PerturbQA}

\textbf{Reviewer jVCL (Weakness, verbatim):}
\begin{quote}
All experiments are within PERTURBQA... Even a smaller secondary biology reasoning task would make the generality claim more credible.
\end{quote}

\vspace{0.4em}
\textbf{Authors (Response, verbatim highlights):}
\begin{quote}
...zero-shot disease-state prediction ... shows rbio nearly matches SCVI ... and clearly outperforms Qwen2.5-3B...
\end{quote}
\begin{quote}
...demonstrating biological reasoning transfer beyond perturbation tasks.
\end{quote}

\vspace{0.4em}
\textbf{Takeaway for students:} To rebut ``narrow benchmark'' critiques:
\begin{itemize}
  \item add a \textbf{cross-domain} experiment (even if smaller)
  \item make it \textbf{zero-shot} to strengthen the generalization claim
\end{itemize}

\end{frame}

\begin{frame}{Reviewer tnsY: Reproducibility and Code Release}

\textbf{Reviewer tnsY (verbatim):}
\begin{quote}
I have some concerns regarding the reproducibility of the results given that the code is not released...
\end{quote}
\begin{quote}
...it is the release of the codebase that will lead me to increase my score.
\end{quote}

\vspace{0.4em}
\textbf{Authors (Response, verbatim highlights):}
\begin{quote}
We have released an anonymous repository ... that reproduces the MLP-verifier experiments and provides an end-to-end example...
\end{quote}
\begin{quote}
...clarified the MLP architecture (activation functions, losses, and hyperparameters)...
\end{quote}

\vspace{0.4em}
\textbf{Takeaway for students:} A reproducibility-focused reviewer can be ``won'' by:
\begin{itemize}
  \item minimal end-to-end repo
  \item architecture + hyperparameters
  \item algorithms and reward computation details
\end{itemize}

\end{frame}

\begin{frame}{Reviewer tnsY: Metrics, Significance, and ``SOTA'' Claims}

\textbf{Reviewer tnsY (verbatim):}
\begin{quote}
...claim of state-of-the-art performance ... when the models are trained with 1/5 of the data sample...
SUMMER appears to perform better on the TNR metric?
\end{quote}
\begin{quote}
...difference in F1-score and the MCC metric does not appear to be statistically significant?
\end{quote}

\vspace{0.4em}
\textbf{Authors (Response, verbatim highlights):}
\begin{quote}
The PerturbQA datasets are class-imbalanced... identifying true positive perturbations is biologically more important...
We therefore emphasize F1, Balanced Accuracy, and MCC...
\end{quote}
\begin{quote}
...SUMMER’s higher TNR reflects a different recall--specificity trade-off rather than superior overall accuracy.
\end{quote}

\vspace{0.4em}
\textbf{Takeaway for students:} When ``SOTA'' is questioned, rebut by:
\begin{itemize}
  \item explicitly arguing \textbf{which metric matches the scientific objective}
  \item explaining trade-offs (TPR vs TNR), not just headline numbers
\end{itemize}

\end{frame}

\begin{frame}{Reviewer AHen: Compute, Scaling, and Baseline Interpretation}

\textbf{Reviewer AHen (verbatim):}
\begin{quote}
...computational resources required are not reported...
\end{quote}
\begin{quote}
GEARS performs surprisingly poorly... Is there any explanation or analysis regarding this?
\end{quote}

\vspace{0.4em}
\textbf{Authors (Response, verbatim highlights):}
\begin{quote}
...Training used ... on 8×H100 GPUs for $\sim$10 days... batch\_size=4, learning\_rate=5e-6...
\end{quote}
\begin{quote}
GEARS’s lower F1 arises from an extreme TNR bias---its high specificity (0.997) is offset by low recall...
\end{quote}

\vspace{0.4em}
\textbf{Takeaway for students:} Always report:
\begin{itemize}
  \item compute budget + steps + key hyperparameters
  \item learning curves / scaling plots
  \item explain surprising baselines (metric mismatch, thresholding effects)
\end{itemize}

\end{frame}

\begin{frame}{Meta: Why This Rebuttal Worked (Writing Guidance)}

\begin{itemize}
  \item They \textbf{matched criticisms with new experiments} (noise, ablations, agreement, cross-domain)
  \item They \textbf{moved from claims to diagnostics} (confidence vs performance, verifier coherence)
  \item They \textbf{fixed reproducibility directly} (repo + algorithms + hyperparams)
  \item They \textbf{clarified evaluation philosophy} (imbalance, objective-aligned metrics)
  \item They added \textbf{stronger baselines} to rule out ``generic RL'' explanations
\end{itemize}

\vspace{0.4em}
\textbf{Teaching takeaway:}
\begin{itemize}
  \item A strong rebuttal is not persuasion---it is \textbf{additional evidence} and \textbf{reduced ambiguity}.
\end{itemize}

\end{frame}

\end{document}
