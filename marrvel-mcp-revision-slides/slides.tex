\documentclass[ 11pt,notheorems,hyperref={pdfauthor=whatever} ]{beamer}


% Copyright (c) 2022 by Lars Spreng
% This work is licensed under the Creative Commons Attribution 4.0 International License.
% To view a copy of this license, visit http://creativecommons.org/licenses/by/4.0/ or send a letter to Creative Commons, PO Box 1866, Mountain View, CA 94042, USA.

%~~~~~~~~~~~~~~~~~~~~~~~~~~~~~~~~~~~~~~~~~~~~~~~~~~~~~~~~~~~~~~~~~~~~~~~~~~~~~~
% Add your packages and commands to this file
%~~~~~~~~~~~~~~~~~~~~~~~~~~~~~~~~~~~~~~~~~~~~~~~~~~~~~~~~~~~~~~~~~~~~~~~~~~~~~~

%~~~~~~~~~~~~~~~~~~~~~~~~~~~~~~~~~~~~~~~~~~~~~~~~~~~~~~~~~~~~~~~~~~~~~~~~~~~~~~
% Fonts
\RequirePackage[scaled]{helvet} % for sans-serif slides

\RequirePackage[utf8]{inputenc}
\RequirePackage[T1]{fontenc}


\usepackage{styles/elegantmacros}
\usefolder{styles}
\usetheme[style=lecture]{elegant}

\newcommand{\makepart}[1]{ % For convenience
\part{#1} \frame{\partpage}
}

%~~~~~~~~~~~~~~~~~~~~~~~~~~~~~~~~~~~~~~~~~~~~~~~~~~~~~~~~~~~~~~~~~~~~~~~~~~~~~~

%~~~~~~~~~~~~~~~~~~~~~~~~~~~~~~~~~~~~~~~~~~~~~~~~~~~~~~~~~~~~~~~~~~~~~~~~~~~~~~
% Figures
\RequirePackage{booktabs}
\RequirePackage{colortbl}
\RequirePackage{ragged2e}
\RequirePackage{caption}
\RequirePackage{subcaption}
\RequirePackage{tabularx}
\RequirePackage{array}
\RequirePackage{multirow}

\newcolumntype{Y}{>{\centering\arraybackslash}X}

%~~~~~~~~~~~~~~~~~~~~~~~~~~~~~~~~~~~~~~~~~~~~~~~~~~~~~~~~~~~~~~~~~~~~~~~~~~~~~~

%~~~~~~~~~~~~~~~~~~~~~~~~~~~~~~~~~~~~~~~~~~~~~~~~~~~~~~~~~~~~~~~~~~~~~~~~~~~~~~
% Figures
\RequirePackage{wrapfig}
\RequirePackage{graphicx}
\RequirePackage{adjustbox}

%~~~~~~~~~~~~~~~~~~~~~~~~~~~~~~~~~~~~~~~~~~~~~~~~~~~~~~~~~~~~~~~~~~~~~~~~~~~~~~

%~~~~~~~~~~~~~~~~~~~~~~~~~~~~~~~~~~~~~~~~~~~~~~~~~~~~~~~~~~~~~~~~~~~~~~~~~~~~~~
% Code listings
\RequirePackage{listings}
\RequirePackage{textcomp}

\lstdefinestyle{terminalstyle}{
  basicstyle=\ttfamily\scriptsize,
  backgroundcolor=\color{black!5},
  frame=single,
  rulecolor=\color{black!30},
  framesep=5pt,
  breaklines=true,
  breakatwhitespace=false,
  showstringspaces=false,
  columns=fullflexible,
  keepspaces=true,
  xleftmargin=5pt,
  xrightmargin=5pt,
  aboveskip=8pt,
  belowskip=4pt,
  keywordstyle=\color{primary}\bfseries,
  commentstyle=\color{Gray}\itshape,
}

\lstdefinestyle{yamlstyle}{
  basicstyle=\ttfamily\scriptsize,
  backgroundcolor=\color{black!3},
  frame=single,
  rulecolor=\color{primary!30},
  framesep=5pt,
  breaklines=true,
  showstringspaces=false,
  columns=fullflexible,
  keepspaces=true,
  xleftmargin=5pt,
  xrightmargin=5pt,
  aboveskip=8pt,
  belowskip=4pt,
  keywordstyle=\color{primary}\bfseries,
  stringstyle=\color{secondary},
  commentstyle=\color{Gray}\itshape,
  morecomment=[l]{\#},
}

%~~~~~~~~~~~~~~~~~~~~~~~~~~~~~~~~~~~~~~~~~~~~~~~~~~~~~~~~~~~~~~~~~~~~~~~~~~~~~~

%~~~~~~~~~~~~~~~~~~~~~~~~~~~~~~~~~~~~~~~~~~~~~~~~~~~~~~~~~~~~~~~~~~~~~~~~~~~~~~
% TikZ for diagrams
\RequirePackage{tikz}
\usetikzlibrary{arrows.meta, positioning, shapes.geometric, calc, fit}

%~~~~~~~~~~~~~~~~~~~~~~~~~~~~~~~~~~~~~~~~~~~~~~~~~~~~~~~~~~~~~~~~~~~~~~~~~~~~~~


\usepackage{tabularx} \usepackage{array} \usepackage{booktabs}


\title[MARRVEL-MCP Revision]{MARRVEL-MCP Revision}

\author[]{Hyun-Hwan Jeong}

\institute{Liu Lab Meeting} \date{February 4, 2026}

\begin{document}

% Title page
{ \setbeamertemplate{footline}{} \begin{frame} \titlepage \end{frame} }
\addtocounter{framenumber}{-1}

%==============================================================================
\section{Overview}
%==============================================================================

\begin{frame}{Revision Summary}

\begin{columns}[T]

\column{0.48\textwidth} \textbf{Reviewer \#1} \hfill \textit{6 comments}

\vspace{0.3em} \small \begin{itemize} \item Positive: timely and important topic
\item Wants stronger positioning vs. existing AI agents \item Requests more
evaluation details and baselines \item Terminology and title wording concerns
\end{itemize}

\column{0.04\textwidth} \rule{0.5pt}{0.55\textheight}

\column{0.48\textwidth} \textbf{Reviewer \#2} \hfill \textit{6 comments}

\vspace{0.3em} \small \begin{itemize} \item Positive: solid engineering, useful
MCP tools \item Wants more honest framing of contributions \item Requests deeper
evaluation and tool analysis \item Concerns about gold standard reliability
\end{itemize}

\end{columns}

\end{frame}

%==============================================================================
\section{Reviewer \#1}
%==============================================================================

\begin{frame}{Reviewer \#1: Overall Comments}

\begin{quote} This manuscript describes MARRVEL-MCP, an AI-agent system that
automatically navigates appropriate biomedical resources to answer rare disease
questions. This is a \textbf{very timely and important topic} for Mendelian
disease discovery in the era of AI. The reviewer has several suggestions for
improving the manuscript. \end{quote}

\vspace{0.6em} \textbf{6 specific comments:} \small \begin{enumerate} \item
Related work on biomedical AI agents \item Clarification of ``context
engineering'' \item Benchmark dataset and evaluation details \item Comparison
with commercial AI tools (GPT, Gemini) \item Title wording clarification \item
Usability improvement evidence \end{enumerate}

\end{frame}

%--- R1.1 ---
\begin{frame}{R1.1: Related Work on Biomedical AI Agents}

\textbf{Reviewer comment:} \begin{quote} \small The authors should consider
including more relevant work on biomedical AI agents (e.g., BioMNI). It would
also be helpful to compare MARRVEL-MCP with existing biomedical AI agents in
terms of architecture, performance, speed, etc., to better highlight the
uniqueness of the proposed system for rare diseases. \end{quote}

\vspace{0.4em} \textbf{Response:}

\small We thank the reviewer for this suggestion. We have expanded the
Background section to include recent biomedical AI agent systems---BioMNI,
RDguru, STELLA, and KGAREVION---and added a systematic comparison table
(Table~1) contrasting architecture, model requirements, data granularity, and
design characteristics. We also clarified MARRVEL-MCP's unique positioning as a
system that enables lightweight, off-the-shelf models (3B--20B) to achieve
expert-level genomic workflows through context engineering rather than
frontier-scale compute.

\end{frame}

%--- R1.1 comparison table ---
\begin{frame}{R1.1: Added Comparison Table}

\textbf{Table 1:} Comparison of MARRVEL-MCP with recent biomedical AI agent frameworks.

\vspace{0.3em}
\centering
\scriptsize
\begin{tabular}{@{} l >{\raggedright}p{0.17\textwidth} >{\raggedright}p{0.22\textwidth} >{\raggedright}p{0.18\textwidth} >{\raggedright\arraybackslash}p{0.22\textwidth} @{}}
\toprule
\textbf{System} & \textbf{Primary Focus} & \textbf{Model Architecture} & \textbf{Data Granularity} & \textbf{Design Characteristics} \\
\midrule
BioMNI & Multimodal reasoning \& dynamic workflow & Agentic module (BioMNI-A1) with frontier models (Claude, GPT-4) & Broad biomedical concepts \& literature & Sophisticated reasoning; requires high-resource infrastructure \\
\addlinespace
STELLA & Self-improving biomedical reasoning & Dynamic tool discovery with BioMNI-A1 backbone & Broad biomedical resources & Adaptive self-evolution; frontier model dependency \\
\addlinespace
KGAREVION & Knowledge-intensive QA via graph verification & LLM verification against static KGs (PrimeKG) & Gene-disease associations from ontologies & Graph-based verification; gene/disease level \\
\addlinespace
RDguru & Rare disease clinical consultation & Knowledge-based conversational interface & Disease-symptom associations & Clinical dialogue support \\
\midrule
\textbf{MARRVEL-MCP} & \textbf{Variant interpretation \& gene analysis} & \textbf{Context-engineered tools with 3B--20B models (no fine-tuning)} & \textbf{Gene/variant-level granularity} & \textbf{Lightweight models via structured DB access; variant-level precision} \\
\bottomrule
\end{tabular}

\end{frame}

%--- R1.1 revised text ---
\begin{frame}{R1.1: Revised Manuscript Text}

\textbf{Added to Background:}

\begin{quote} \scriptsize More recently, specialized biomedical AI agents have
emerged as a transformative paradigm for clinical applications, moving beyond
passive question-answering to active problem-solving through autonomous tool use
and multi-step reasoning [Gao 2024]. These systems include BioMNI for multimodal
reasoning through literature-mining dynamic workflow composition [Huang 2025],
RDguru for rare disease diagnosis through knowledge-based question answering and
clinical consultation support [Yang 2025], STELLA for self-improving biomedical
reasoning through dynamic tool discovery and evolving reasoning templates [Jin
2025], and KGAREVION for knowledge-intensive medical question answering through
knowledge graph verification of LLM-generated information [Su 2024].

\vspace{0.3em} We systematically compare MARRVEL-MCP with these frameworks (see
Table~X) and illustrate how our architecture enables \textbf{lightweight,
off-the-shelf models} to match the utility of high-resource generalist agents
through specialized context engineering. While systems like BioMNI and STELLA
demonstrate impressive reasoning, their agentic modules (e.g., BioMNI-A1) depend
on computationally intensive frontier models (e.g., Claude, GPT-4) to
orchestrate complex tasks, limiting their deployability in local or low-resource
environments. Conversely, approaches like KGAREVION rely on static knowledge
graphs (PrimeKG) [Chandak 2023] that lack the \textit{variant-level
granularity}---e.g., specific genomic coordinates and allele
frequencies---required for pathogenicity diagnosis.

\vspace{0.3em} MARRVEL-MCP breaks this trade-off between model accessibility and
analytical depth. By defining a rigorous ``context engineering'' environment, we
establish a new paradigm: \textbf{enabling lightweight, off-the-shelf models
(3B--20B) to autonomously execute expert-level genomic workflows without the
need for massive scale or specialized fine-tuning}. This approach effectively
offloads the requirement for ``intelligence'' from the model parameters to the
tool environment, allowing small local agents to perform with the precision of
frontier-scale systems. \end{quote}

\end{frame}

%--- R1.2 ---
\begin{frame}{R1.2: Clarification of ``Context Engineering''}

\textbf{Reviewer comment:} \begin{quote} \small The term \emph{context
engineering} has been used to describe the next generation of RAG systems. Its
definition typically extends beyond selecting appropriate input resources, which
appears to be the main definition used in this study. Please ensure that the
usage in the manuscript is precise and consistent with existing terminology.
\end{quote}

\vspace{0.4em} \textbf{Response:}

\small We thank the reviewer for raising this important point. We have revised
the manuscript across the Abstract, Background, and Discussion sections to adopt
the broader, formally recognized definition of context engineering [Mei 2025].
We now explicitly distinguish context engineering from conventional RAG and from
general-purpose agent frameworks, and clarify that our work focuses specifically
on tool-mediated autonomous analysis as one instantiation within the broader
context engineering taxonomy. The term ``tool-augmented context engineering'' is
now used consistently throughout.

\end{frame}

%--- R1.2 revised abstract ---
\begin{frame}{R1.2: Revised Abstract}

\textbf{Revised in Abstract:}

\begin{quote} \scriptsize This work demonstrates the impact of
\textbf{tool-augmented context engineering}---the deliberate design of
domain-aware tool environments and structured information scaffolding through
executable function interfaces---in reshaping the role of model scale in
genomics. MARRVEL-MCP equips LLMs with 39 tools spanning gene and variant
utilities, pathogenicity databases, phenotype resources, expression atlases,
ortholog data, and literature APIs. Without hard-coded pipelines, LLMs
autonomously infer workflows, performing named-entity recognition, identifier
normalization, and multi-database synthesis from narrative queries. \end{quote}

\end{frame}

%--- R1.2 revised background ---
\begin{frame}{R1.2: Revised Background}

\textbf{Added to Background:}

\begin{quote} \scriptsize Context-engineering, as recently formalized,
encompasses the systematic optimization of information payloads for LLMs through
the structured assembly of multiple context components, including instructions,
knowledge, tools, memory, and system state [Mei 2025]. These multiple context
components fundamentally distinguish context engineering from conventional
retrieval-augmented generation (RAG). While conventional RAG retrieves relevant
text passages to augment prompts, it remains constrained by static workflows,
lacks executable tools and structured processes, and provides insufficient
adaptability for multi-step reasoning and complex task management [Singh 2025].

\vspace{0.2em} Context-engineering also differs from general-purpose agent
frameworks such as AutoGPT and ReAct, which provide planning scaffolds but lack
the domain-specific tools and constraints necessary for biological reasoning. A
general agent might generate a plan like ``Step 1: Search for variant
pathogenicity. Step 2: Summarize findings''---but without access to ClinVar's
API and knowledge of HGVS notation, it cannot execute this plan reliably.

\vspace{0.2em} Applying these principles, our work focuses specifically
on \textbf{tool-mediated autonomous analysis}: we provide LLMs with executable
function interfaces to curated databases, enabling them to discover and select
appropriate analytical pathways rather than following predefined workflows. By
designing tool interfaces that communicate not just \textit{what} they compute
but \textit{when} and \textit{why} they should be used in biological
investigation, we enable the LLM to autonomously navigate complex analytical
decisions analogous to those made by experienced computational biologists.
\end{quote}

\end{frame}

%--- R1.2 revised discussion 1 ---
\begin{frame}{R1.2: Revised Discussion (1/2)}

\textbf{Added to Discussion:}

\begin{quote} \scriptsize MARRVEL-MCP represents a specific instantiation of
context engineering focused on tool-mediated database access. Recent taxonomies
of context engineering [Mei 2025] identify several complementary approaches to
optimizing LLM information payloads: (1) RAG systems that retrieve and re-rank
text documents, (2) memory systems that maintain conversation state, (3)
tool-integrated reasoning that provides executable functions, and (4)
multi-agent orchestration for complex workflows. Our work focuses on
``tool-integrated reasoning''---demonstrating that context quality can
compensate for model scale, regardless of whether the context originates from
retrieved documents or structured database queries.

\vspace{0.2em} MARRVEL-MCP demonstrates that structured tool environments can
reshape the relationship between model scale and genomic task performance. A
20B-parameter model with MARRVEL-MCP achieved 95\% accuracy compared to 33\%
without tools---matching systems 5--10$\times$ larger and approaching the
performance of state-of-the-art proprietary models. This pattern held
consistently: lightweight models with appropriate context outperformed much
larger systems operating without tool access.

\vspace{0.2em} This effect stems from context engineering---encoding domain
constraints, nomenclature standards, coordinate systems, and provenance metadata
directly into the tool environment. The rs193922679 example crystallizes this
principle: Haiku with MARRVEL-MCP delivered structured dbNSFP predictions
formatted for clinical review by autonomously inferring the correct workflow
(rsID conversion, coordinate-based query, tabular summary), while a much larger
model without tools provided only narrative web search results. \end{quote}

\end{frame}

%--- R1.2 revised discussion 2 ---
\begin{frame}{R1.2: Revised Discussion (2/2)}

\textbf{Discussion (continued):}

\begin{quote} \scriptsize The key insight generalizes: the systematic design of
information structure and access patterns enables smaller models to achieve
performance comparable to frontier systems on domain-specific tasks. Future
genomic assistants might benefit from combining these approaches: using RAG for
literature synthesis and clinical note analysis, while employing tool-based
systems like MARRVEL-MCP for structured variant annotation and database queries.
Such hybrid architectures would leverage the complementary strengths of
different context engineering strategies---the semantic flexibility of retrieval
with the computational precision of structured tools. \end{quote}

\vspace{0.3em} \textbf{Revised Conclusion:}

\begin{quote} \scriptsize This work demonstrates that tool-augmented context
engineering---the systematic design of domain-specific function interfaces and
structured information flows---can compensate for limited model capacity in
specialized domains. By optimizing how genomic information is accessed and
formatted rather than relying solely on model scale, MARRVEL-MCP enables
lightweight models to achieve accuracy comparable to frontier systems on variant
interpretation tasks. This finding generalizes beyond genomics: context
engineering, whether through document retrieval, tool integration, or hybrid
approaches, represents a fundamental strategy for building reliable, efficient
domain-specific AI systems. Future work should explore integration of multiple
context engineering modalities to leverage their complementary strengths.
\end{quote}

\end{frame}

%--- R1.3 ---
\begin{frame}{R1.3: Benchmark Dataset and Evaluation Details}

\textbf{Reviewer comment:} \begin{quote} \small Please provide additional
details about the benchmark dataset and evaluation pipeline. What criteria were
used to define ``answerable'' questions? Were answers manually generated by
experts? When Claude Sonnet was used as an external judge, did it compare with
expert-generated reference answers, or rely on its own internal knowledge?
\end{quote}

\vspace{0.4em} \textbf{Response:}

\small We thank the reviewer for this question. We have added a new
Implementation subsection detailing the benchmark construction and evaluation
pipeline. The benchmark of 100 questions was manually constructed and
categorized into baseline, single-tool, and multi-tool queries. Expected answers
were curated by team members with expertise in molecular genetics. Claude Sonnet
4 was used as an automated judge that directly compared LLM responses against
the human-curated expected answers---not its own internal knowledge. For
time-sensitive questions, automated judgments were supplemented with manual
review. The evaluation code is publicly available.

\end{frame}

%--- R1.3 revised text: benchmark ---
\begin{frame}{R1.3: Revised Implementation -- Benchmark Dataset}

\textbf{Added to Implementation:}

\begin{quote} \scriptsize We manually constructed a benchmark dataset of 100
questions designed to evaluate variant interpretation capabilities answerable
using MARRVEL data sources. The benchmark encompassed diverse query types
encountered in clinical genetics, from basic gene function lookups to complex
multi-source queries requiring integration of variant databases, model organism
phenotypes, protein interactions, and recent publications. Questions were
categorized into two types based on the complexity of tool calling usage: (1)
baseline queries answerable without MARRVEL-MCP using pre-trained knowledge, (2)
single-tool queries requiring a single MARRVEL-MCP tool call, and (3) multi-tool
queries requiring information synthesis across multiple sequential tool calls
and data sources.

\vspace{0.2em} For each question, expected answers were manually curated by
members of our team with expertise in molecular genetics and variant
interpretation. These reference answers were created by systematically querying
MARRVEL directly and synthesizing the returned information into concise,
accurate responses. Each reference answer included key details that served as
critical criteria for evaluating correctness. For time-sensitive
questions---such as those involving recent PubMed publications---reference
answers cannot be predetermined, so we manually reviewed each LLM response
against the current state of the data sources at the time of evaluation.
\end{quote}

\end{frame}

%--- R1.3 revised text: evaluation pipeline ---
\begin{frame}{R1.3: Revised Implementation -- Evaluation Pipeline}

\textbf{Added to Implementation:}

\begin{quote} \scriptsize The evaluation pipeline consisted of three main
stages: response generation, automated judging with manual review, and
statistical analysis.

\vspace{0.2em} \textbf{Response Generation:} Each of the nine LLMs was tested
under two conditions: (1) baseline without MARRVEL-MCP access, and (2)
tool-enabled with full access to all 39 tools. All API calls were logged to
ensure reproducibility and to analyze tool usage patterns.

\vspace{0.2em} \textbf{Automated Judging with Manual Review:} We employed Claude
Sonnet 4 as an automated judge. For each test case, the judge was presented
with: (1) the original question, (2) the human-curated expected answer, and (3)
the LLM-generated response. \textbf{Crucially, the judge directly compared the
LLM response with the human-curated expected answer rather than relying on its
own internal knowledge.} The judging prompt instructed Claude Sonnet 4 to verify
that all key facts were present, check for factual contradictions, allow
different phrasings, mark hallucinated information as incorrect, and account for
multi-step reasoning. For dynamic data sources, automated judgments were
supplemented with manual review.

\vspace{0.2em} \textbf{Pass Rate Calculation:} For each model and condition, we
calculated the pass rate as the proportion of questions where the LLM response
was judged to agree with the human-curated expected answer. \end{quote}

\end{frame}

%--- R1.3 revised text: result section ---
\begin{frame}{R1.3: Revised Result Section}

\textbf{Added to Results:}

\begin{quote} \scriptsize Equipping lightweight models with MARRVEL-MCP
substantially improved their accuracy on variant interpretation tasks, enabling
smaller models to match or exceed the baseline performance of much larger
systems. We evaluated this effect across nine LLMs ranging from 3B to 235B
parameters using a benchmark of 45 expert-curated questions answerable using
MARRVEL data. Each model was tested in two conditions: without MARRVEL-MCP
access (baseline) and with full access to all 39 tools. Responses were evaluated
by comparing them to an expert-generated truth set of answers. Claude Sonnet 4
was used to judge if the LLM response agreed with the expert response, with pass
rate defined as the proportion of questions judged correct. \end{quote}

\end{frame}

%--- R1.4 ---
\begin{frame}{R1.4: Comparison with Commercial AI Tools}

\textbf{Reviewer comment:}
\begin{quote} The authors should include the
performance of state-of-the-art commercial AI tools such as GPT and Gemini on
the benchmark dataset. If these systems already achieve strong performance,
please justify the added value of MARRVEL-MCP (e.g., smaller models, lower cost,
domain customization, transparency, or reproducibility). \end{quote}

\vspace{0.6em} \textbf{Action items:}
\begin{itemize}
\item Benchmark GPT-4o / GPT-o1 and Gemini on evaluation set
\item Report performance comparison table
\item Justify added value: domain tools, cost, transparency, reproducibility
\end{itemize}

\end{frame}

%--- R1.5 ---
\begin{frame}{R1.5: Title Wording Clarification}

\textbf{Reviewer comment:} \begin{quote} \small Is MARRVEL-MCP primarily a
question-answering system, or does it function as a chatbot that supports
multi-turn conversations? I suggest replacing ``NATURAL-LANGUAGE
QUERY-TO-RESPONSE INTERFACE'' in the title with ``Question-Answering System'' or
``Chatbot'' to improve clarity and precision. \end{quote}

\vspace{0.4em} \textbf{Response:}

\small We appreciate this suggestion. We have revised the title to:
\textbf{``MARRVEL-MCP: An Agentic Interface for Mendelian Disease Discovery via
Tool-Augmented Context Engineering.''} We chose ``Agentic Interface'' rather
than ``Question-Answering System'' or ``Chatbot'' because MARRVEL-MCP is not a
simple conversational bot---it is a system where the LLM autonomously selects
and orchestrates 39 domain-specific tools to perform multi-step genomic
analysis. The term ``interface'' emphasizes its role as a bridge between the
user and structured biological databases, while ``agentic'' captures the
autonomous tool-use capability that distinguishes it from passive QA systems.
This framing highlights both the interaction design and the underlying tool
orchestration that define the system's contribution.

\end{frame}

%--- R1.6 ---
\begin{frame}{R1.6: Usability Improvement Evidence}

\textbf{Reviewer comment:}
\begin{quote} The authors have noted that
user-friendliness is one of the motivations for developing MARRVEL-MCP. However,
is there any evidence showing that MARRVEL-MCP provides outputs that are more
beneficial to users? If there has not been a formal user study, the authors
should consider addressing this point briefly in the discussion section.
\end{quote}

\vspace{0.6em} \textbf{Action items:}
\begin{itemize}
\item Acknowledge lack of formal user study (if applicable)
\item Add discussion paragraph on usability benefits
\item Consider citing indirect evidence or anecdotal feedback
\item Mention user study as future work
\end{itemize}

\end{frame}

%==============================================================================
\section{Reviewer \#2}
%==============================================================================

\begin{frame}{Reviewer \#2: Overall Comments}

\begin{quote} \small I think this paper is on the right track, but right now it
feels more like a \textbf{solid engineering effort} than a big conceptual leap.
The MCP tools are genuinely useful, but the manuscript could be more honest
about what is new versus what is just well-integrated. The MCP layers the
authors built are probably the \textbf{most valuable part} of the work. They
make it much easier for LLMs to interact with biological databases like dbNSFP,
which is great for the community. The tooling infrastructure is the most
compelling contribution, but the \textbf{system-level claims need stronger
justification} and clearer evidence. \end{quote}

\vspace{0.6em} \textbf{6 specific comments:} \small \begin{enumerate} \item
Context engineering already in most agent systems \item ``Hard-coded'' vs.
non-hard-coded clarification \item Evaluation is thin; hand-picked examples
\item Which MCP tools help the most? \item Error types count mismatch (Sec. 3.3)
\item Gold standard reliability concern \end{enumerate}

\end{frame}

%--- R2.1 ---
\begin{frame}{R2.1: Context Engineering in Agent Systems}

\textbf{Reviewer comment:}
\begin{quote} In most agent systems, context
engineering is already part of the architecture, so the comparison in the intro
is confusing. \end{quote}

\vspace{0.4em} \textbf{Response:}

\small We thank the reviewer for this observation. We have substantially revised
the Background to clarify what distinguishes our use of context engineering from
standard agent-system context handling. We now explicitly distinguish context
engineering from (1) conventional RAG, which lacks executable tools and
structured processes, and (2) general-purpose agent frameworks such as AutoGPT
and ReAct, which provide planning scaffolds but lack domain-specific tools and
constraints. We define our contribution as \textbf{tool-mediated autonomous
analysis}---providing LLMs with executable function interfaces that communicate
not just \textit{what} they compute but \textit{when} and \textit{why} they
should be used in biological investigation. This reframes our novelty: not
context engineering per se, but domain-specific context engineering that embeds
biological reasoning constraints into the tool environment.

\end{frame}

%--- R2.1 revised text: background ---
\begin{frame}{R2.1: Revised Background (1/2)}

\textbf{Added to Background:}

\begin{quote} \scriptsize
Context-engineering, as recently formalized, encompasses the systematic
optimization of information payloads for LLMs through the structured assembly
of multiple context components, including instructions, knowledge, tools,
memory, and system state [Mei 2025]. These multiple context components
fundamentally distinguish context engineering from conventional
retrieval-augmented generation (RAG). While conventional RAG retrieves relevant
text passages to augment prompts, it remains constrained by static workflows,
lacks executable tools and structured processes, and provides insufficient
adaptability for multi-step reasoning and complex task management [Singh 2025].

\vspace{0.2em}
Context-engineering also differs from general-purpose agent frameworks such as
AutoGPT and ReAct, which provide planning scaffolds but lack the domain-specific
tools and constraints necessary for biological reasoning. A general agent might
generate a plan like ``Step 1: Search for variant pathogenicity. Step 2:
Summarize findings''---but without access to ClinVar's API and knowledge of HGVS
notation, it cannot execute this plan reliably.
\end{quote}

\end{frame}

%--- R2.1 revised text: background 2 ---
\begin{frame}{R2.1: Revised Background (2/2)}

\textbf{Background (continued):}

\begin{quote} \scriptsize
Applying these principles, our work focuses specifically on \textbf{tool-mediated
autonomous analysis}: we provide LLMs with executable function interfaces to
curated databases, enabling them to discover and select appropriate analytical
pathways rather than following predefined workflows. This approach exemplifies
how context-engineering principles---optimizing which information is available
and how it is structured---can be instantiated through domain-specific tooling
that exposes biological purpose alongside computational function. By designing
tool interfaces that communicate not just \textit{what} they compute but
\textit{when} and \textit{why} they should be used in biological investigation,
we enable the LLM to autonomously navigate complex analytical decisions
analogous to those made by experienced computational biologists. This creates a
system in which domain knowledge is embedded in both the tool environment and
the contextual framework that guides tool selection, enabling reliable
computational workflows while preserving adaptability in biological
problem-solving.
\end{quote}

\end{frame}

%--- R2.2 ---
\begin{frame}{R2.2: Hard-Coded vs. Non-Hard-Coded}

\textbf{Reviewer comment:} \begin{quote} Authors emphasized not ``hard-coded'';
I partially agree. But given many coding is implemented in the MCP layer
(providing interfaces), I won't say it is completely non hard-coded. The
improvement of MCP over the baseline is because of the ``hard-coding'' that
exposes the abstract layer usable by LLM agents. A better way to describe: MCP
is certainly hard-coded, but the workflow is not. \end{quote}

\vspace{0.6em} \textbf{Action items:} \begin{itemize} \item Adopt nuanced
framing: MCP tools are hard-coded interfaces, \\ but the
\textbf{workflow/reasoning} is not hard-coded \item Revise manuscript language
accordingly \end{itemize}

\end{frame}

%--- R2.3 ---
\begin{frame}{R2.3: Evaluation Concerns}

\textbf{Reviewer comment:} \begin{quote} \small The current evaluation is a bit
thin. The examples in Section 3.1 are hand-picked and likely biased. For the 45
questions, it's not clear which ones actually need MCP tools and which ones
don't. I'd really like to see cases when questions do not need MCP tools, how it
performed comparing to the baseline. \end{quote}

\vspace{0.6em} \textbf{Action items:}
\begin{itemize}
\item Bump benchmark from 45 to 100 questions
\item Finalizing the benchmark revision
\item Check the list of questionnaires that would be answered by LLM or not
\end{itemize}

\end{frame}

%--- R2.4 ---
\begin{frame}{R2.4: Tool Usage Analysis}

\textbf{Reviewer comment:} \begin{quote} It would be helpful to know which MCP
tools actually help the most? How many tool calls are needed per question?
Whether performance drops as tool usage increases? \end{quote}

\vspace{0.6em} \textbf{Action items:} \begin{itemize} \item Report per-tool
contribution to answer quality \item Analyze distribution of tool calls per
question \item Investigate performance vs. number of tool calls \item Add tool
usage breakdown figure or table \end{itemize}

\end{frame}

%--- R2.5 ---
\begin{frame}{R2.5: Error Types Count Mismatch}

\textbf{Reviewer comment:} \begin{quote} \small There's a small error where the
paper says there are three error types but only lists two. (Section 3.3: ``We
categorized errors into three types based on their underlying cause.'')
\end{quote}

\vspace{0.4em} \textbf{Response:}

\small We thank the reviewer for catching this error. We have corrected the
count in Section 3.3 to match the number of error types actually listed. We have
also conducted a more careful proofreading pass across the entire manuscript to
ensure consistency between numerical references and their corresponding content
throughout.

\end{frame}

%--- R2.6 ---
\begin{frame}{R2.6: Gold Standard Reliability}

\textbf{Reviewer comment:} \begin{quote} As LLMs become stronger, the
reliability of purely human-curated ``gold standard'' answers becomes less
clear. It would be helpful to know whether the authors manually reviewed
LLM-generated ``error'' answers to confirm they are truly incorrect, rather than
alternative valid answers. \end{quote}

\vspace{0.6em} \textbf{Action items:} \begin{itemize} \item Manually review LLM
answers marked as ``errors'' \item Report how many ``errors'' were actually
valid alternative answers \item Discuss limitations of human-curated gold
standards \item Consider inter-annotator agreement or adjudication process
\end{itemize}

\end{frame}

\end{document}
