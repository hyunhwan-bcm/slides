%~~~~~~~~~~~~~~~~~~~~~~~~~~~~~~~~~~~~~~~~~~~~~~~~~~~~~~~~~~~~~~~~~~~~~~~~~~~~~~
% Updating the Liu Lab Website: A Hands-On Tutorial
% Teaches CLI-based GitHub PR workflow, then explains the WHY
%~~~~~~~~~~~~~~~~~~~~~~~~~~~~~~~~~~~~~~~~~~~~~~~~~~~~~~~~~~~~~~~~~~~~~~~~~~~~~~

\documentclass[
11pt,notheorems,hyperref={pdfauthor=whatever}
]{beamer}


% Copyright (c) 2022 by Lars Spreng
% This work is licensed under the Creative Commons Attribution 4.0 International License.
% To view a copy of this license, visit http://creativecommons.org/licenses/by/4.0/ or send a letter to Creative Commons, PO Box 1866, Mountain View, CA 94042, USA.

%~~~~~~~~~~~~~~~~~~~~~~~~~~~~~~~~~~~~~~~~~~~~~~~~~~~~~~~~~~~~~~~~~~~~~~~~~~~~~~
% Add your packages and commands to this file
%~~~~~~~~~~~~~~~~~~~~~~~~~~~~~~~~~~~~~~~~~~~~~~~~~~~~~~~~~~~~~~~~~~~~~~~~~~~~~~

%~~~~~~~~~~~~~~~~~~~~~~~~~~~~~~~~~~~~~~~~~~~~~~~~~~~~~~~~~~~~~~~~~~~~~~~~~~~~~~
% Fonts
\RequirePackage[scaled]{helvet} % for sans-serif slides

\RequirePackage[utf8]{inputenc}
\RequirePackage[T1]{fontenc}


\usepackage{styles/elegantmacros}
\usefolder{styles}
\usetheme[style=lecture]{elegant}

\newcommand{\makepart}[1]{ % For convenience
\part{#1} \frame{\partpage}
}

%~~~~~~~~~~~~~~~~~~~~~~~~~~~~~~~~~~~~~~~~~~~~~~~~~~~~~~~~~~~~~~~~~~~~~~~~~~~~~~

%~~~~~~~~~~~~~~~~~~~~~~~~~~~~~~~~~~~~~~~~~~~~~~~~~~~~~~~~~~~~~~~~~~~~~~~~~~~~~~
% Figures
\RequirePackage{booktabs}
\RequirePackage{colortbl}
\RequirePackage{ragged2e}
\RequirePackage{caption}
\RequirePackage{subcaption}
\RequirePackage{tabularx}
\RequirePackage{array}
\RequirePackage{multirow}

\newcolumntype{Y}{>{\centering\arraybackslash}X}

%~~~~~~~~~~~~~~~~~~~~~~~~~~~~~~~~~~~~~~~~~~~~~~~~~~~~~~~~~~~~~~~~~~~~~~~~~~~~~~

%~~~~~~~~~~~~~~~~~~~~~~~~~~~~~~~~~~~~~~~~~~~~~~~~~~~~~~~~~~~~~~~~~~~~~~~~~~~~~~
% Figures
\RequirePackage{wrapfig}
\RequirePackage{graphicx}
\RequirePackage{adjustbox}

%~~~~~~~~~~~~~~~~~~~~~~~~~~~~~~~~~~~~~~~~~~~~~~~~~~~~~~~~~~~~~~~~~~~~~~~~~~~~~~

%~~~~~~~~~~~~~~~~~~~~~~~~~~~~~~~~~~~~~~~~~~~~~~~~~~~~~~~~~~~~~~~~~~~~~~~~~~~~~~
% Code listings
\RequirePackage{listings}
\RequirePackage{textcomp}

\lstdefinestyle{terminalstyle}{
  basicstyle=\ttfamily\scriptsize,
  backgroundcolor=\color{black!5},
  frame=single,
  rulecolor=\color{black!30},
  framesep=5pt,
  breaklines=true,
  breakatwhitespace=false,
  showstringspaces=false,
  columns=fullflexible,
  keepspaces=true,
  xleftmargin=5pt,
  xrightmargin=5pt,
  aboveskip=8pt,
  belowskip=4pt,
  keywordstyle=\color{primary}\bfseries,
  commentstyle=\color{Gray}\itshape,
}

\lstdefinestyle{yamlstyle}{
  basicstyle=\ttfamily\scriptsize,
  backgroundcolor=\color{black!3},
  frame=single,
  rulecolor=\color{primary!30},
  framesep=5pt,
  breaklines=true,
  showstringspaces=false,
  columns=fullflexible,
  keepspaces=true,
  xleftmargin=5pt,
  xrightmargin=5pt,
  aboveskip=8pt,
  belowskip=4pt,
  keywordstyle=\color{primary}\bfseries,
  stringstyle=\color{secondary},
  commentstyle=\color{Gray}\itshape,
  morecomment=[l]{\#},
}

%~~~~~~~~~~~~~~~~~~~~~~~~~~~~~~~~~~~~~~~~~~~~~~~~~~~~~~~~~~~~~~~~~~~~~~~~~~~~~~

%~~~~~~~~~~~~~~~~~~~~~~~~~~~~~~~~~~~~~~~~~~~~~~~~~~~~~~~~~~~~~~~~~~~~~~~~~~~~~~
% TikZ for diagrams
\RequirePackage{tikz}
\usetikzlibrary{arrows.meta, positioning, shapes.geometric, calc, fit}

%~~~~~~~~~~~~~~~~~~~~~~~~~~~~~~~~~~~~~~~~~~~~~~~~~~~~~~~~~~~~~~~~~~~~~~~~~~~~~~
 % Loads packages and some defined commands


\title[
Updating Our Lab Website
]{Updating the Liu Lab Website \\ via GitHub Pull Requests}


\author[
Liu Lab
]{
    Hyun-Hwan Jeong
}

\institute{Liu Lab Meeting}
\date{\today}

\begin{document}

% Generate title page
{
\setbeamertemplate{footline}{}
\begin{frame}
  \titlepage
\end{frame}
}
\addtocounter{framenumber}{-1}


%==============================================================================
% MOTIVATION: THE WHY
%==============================================================================

\begin{frame}{Problem: Centralized Website Maintenance}
  \begin{columns}[T]
    \begin{column}{0.55\textwidth}
      \textbf{\color{secondary} The old way (WordPress):}
      \begin{itemize}
        \item Server-based, needs constant maintenance
        \item One person maintains the website
        \item Everyone sends updates via email/Slack
        \item Bottleneck --- slow updates
        \item Single point of failure
      \end{itemize}
    \end{column}
    \begin{column}{0.42\textwidth}
      \textbf{\color{primary} The new way (Static Site):}
      \begin{itemize}
        \item Compile once, add items gradually
        \item \textbf{You} update your own info via PR
        \item No server maintenance needed
        \item No bottleneck
        \item Shared responsibility
      \end{itemize}
    \end{column}
  \end{columns}

  \vspace{1em}
  \begin{exampleblock}{Goal}
    Each lab member owns and maintains their own profile on the lab website.
  \end{exampleblock}
\end{frame}


%==============================================================================
\section{Why: Learn the GitHub Ecosystem}

\begin{frame}{The GitHub Workflow Is Everywhere}
  The fork $\rightarrow$ branch $\rightarrow$ PR workflow you just learned is \textbf{the standard} in:

  \vspace{0.5em}
  \begin{itemize}
    \item \textbf{Open-source software} --- contributing to any project on GitHub
    \item \textbf{Industry} --- how teams collaborate at Google, Meta, startups
    \item \textbf{Research} --- collaborative papers, shared analysis pipelines
    \item \textbf{Data science} --- reproducible workflows and version control
  \end{itemize}

  \vspace{1em}
  \begin{block}{Transferable skills}
    By updating the lab website, you practice the \textbf{exact same workflow} used to contribute to projects like \texttt{scikit-learn}, \texttt{pandas}, \texttt{pytorch}, etc.
  \end{block}
\end{frame}


\begin{frame}{Key Concepts You're Learning}

  \begin{center}
  \begin{tikzpicture}[
    node distance=1.2cm and 2.5cm,
    concept/.style={
      rectangle, rounded corners=4pt,
      draw=primary, fill=primary!8,
      text width=3cm, align=center,
      minimum height=1cm,
      font=\small\bfseries
    },
    arrow/.style={-{Stealth[length=6pt]}, thick, primary!70},
  ]
    \node[concept] (fork)    {Fork};
    \node[concept, right=of fork]  (branch)  {Branch};
    \node[concept, right=of branch] (commit)  {Commit};
    \node[concept, below=of commit] (push)    {Push};
    \node[concept, below=of branch] (pr)      {Pull Request};
    \node[concept, below=of fork]   (merge)   {Merge};

    \draw[arrow] (fork) -- (branch);
    \draw[arrow] (branch) -- (commit);
    \draw[arrow] (commit) -- (push);
    \draw[arrow] (push) -- (pr);
    \draw[arrow] (pr) -- (merge);
  \end{tikzpicture}
  \end{center}

  \vspace{0.5em}
  Each step maps to a \textbf{real-world collaboration pattern}: isolation (branch), review (PR), integration (merge).
\end{frame}


%==============================================================================
% PART 1: THE HOW
%==============================================================================

\section{How: Prerequisites}

\begin{frame}[fragile]{What You Need}
  \begin{columns}[T]
    \begin{column}{0.48\textwidth}
      \textbf{\color{primary} Tools}
      \begin{itemize}
        \item A \textbf{GitHub account}
        \item \texttt{git} installed on your machine
        \item \texttt{gh} --- the GitHub CLI (optional)
        \item A text editor (VS Code, nano, vim, etc.)
      \end{itemize}
    \end{column}
    \begin{column}{0.48\textwidth}
      \textbf{\color{primary} Knowledge (optional)}
      \begin{itemize}
        \item Basic terminal / command line
        \item Markdown syntax
        \item YAML frontmatter (we'll cover this)
      \end{itemize}
    \end{column}
  \end{columns}

  \vspace{1em}
  \begin{block}{Install GitHub CLI}
    \begin{lstlisting}[style=terminalstyle]
# macOS
brew install gh

# Windows
winget install --id GitHub.cli

# Then authenticate
gh auth login
    \end{lstlisting}
  \end{block}
\end{frame}


%==============================================================================
\section{How: Fork, Clone, and Branch}

\begin{frame}[fragile]{Step 1: Fork the Repository}
  Go to \textcolor{primary}{\texttt{github.com/LiuzLab/Liuzlab.github.io}} and click \textbf{Fork}

  \vspace{0.5em}
  Or use the CLI:

  \begin{lstlisting}[style=terminalstyle]
# Fork the repository to your GitHub account
gh repo fork LiuzLab/Liuzlab.github.io --clone
  \end{lstlisting}

  This will:
  \begin{enumerate}
    \item Create a copy (\textbf{fork}) under your GitHub account
    \item Clone it to your local machine
    \item Set up \texttt{origin} (your fork) and \texttt{upstream} (the lab repo)
  \end{enumerate}
\end{frame}


\begin{frame}[fragile]{Step 2: Create a Branch}
  \textbf{Never work directly on \texttt{main}.} Always create a branch:

  \begin{lstlisting}[style=terminalstyle]
# Navigate into the repo
cd Liuzlab.github.io

# Make sure you're up to date
git checkout main
git pull upstream main

# Create a new branch for your changes
git checkout -b update/your-name-info
  \end{lstlisting}

  \vspace{0.5em}
  \begin{alertblock}{Branch naming convention}
    Use descriptive names: \texttt{update/jane-smith-info}, \texttt{fix/typo-in-bio}, \texttt{add/new-member-alex}
  \end{alertblock}
\end{frame}


%==============================================================================
\section{How: Edit Your Member Profile}

\begin{frame}[fragile]{The Website Structure (Astro v6)}
  \begin{columns}[T]
    \begin{column}{0.45\textwidth}
      The lab website is built with \textbf{Astro} --- a modern static site generator.

      \vspace{0.5em}
      Key directories:
      \begin{itemize}
        \item \texttt{src/content/team/} --- \textbf{Member profiles}
        \item \texttt{src/assets/} --- Photos \& images
        \item \texttt{src/pages/} --- Page layouts
      \end{itemize}

      \vspace{0.5em}
      \textbf{\color{primary} File naming convention:}

      \texttt{\color{secondary}\{2-3 letter role\}-\{first\}-\{last\}.md}

      \vspace{0.3em}
      {\scriptsize
      \begin{tabular}{l l}
        \texttt{prof-} & Professor \\
        \texttt{ap-} & Assistant Professor \\
        \texttt{pd-} & Postdoc \\
        \texttt{phd-} & PhD Student \\
        \texttt{ms-} & Master Student \\
        \texttt{ug-} & Undergraduate \\
        \texttt{ae-} & Alumni/Emeritus \\
      \end{tabular}
      }
    \end{column}
    \begin{column}{0.52\textwidth}
      \begin{lstlisting}[style=terminalstyle, title={\scriptsize Repository layout}]
Liuzlab.github.io/
  src/
    content/
      team/
        prof-zhandong-liu.md
        ap-hyun-hwan-jeong.md
        phd-emily-chen.md
        ae-johnathan-jia.md
        ...
    assets/
      zhandong-liu.jpg
      hyun-hwan-jeong.jpg
      ...
    pages/
      ...
      \end{lstlisting}
    \end{column}
  \end{columns}
\end{frame}


\begin{frame}[fragile]{Step 3: Edit Your Profile --- Real Example}
  \begin{columns}[T]
    \begin{column}{0.48\textwidth}
      \textbf{\color{primary} YAML frontmatter} (metadata):
      \begin{lstlisting}[style=yamlstyle, basicstyle=\ttfamily\tiny]
---
name: "Hyun-Hwan Jeong"
role: "Assistant Professor"
title: ["Assistant Professor"]
avatar: "../../assets/hyun-hwan-jeong.jpg"
bio: "Computational biology and
  bioinformatics tool development"
email: "hyun-hwan.jeong@bcm.edu"
linkedin: "https://www.linkedin.com/
  in/hyunhwan-jeong/"
github: "https://github.com/hyunhwan-bcm"
googleScholar: "https://scholar.google.
  com/citations?user=RHsMwPUAAAAJ"
weight: 2
---
      \end{lstlisting}
    \end{column}
    \begin{column}{0.48\textwidth}
      \textbf{\color{primary} What to fill in:}

      \vspace{0.3em}
      {\scriptsize
      \begin{tabularx}{\textwidth}{l X}
        \texttt{name} & \textbf{Your full name} \\[2pt]
        \texttt{role} & Pick from the valid list \\[2pt]
        \texttt{avatar} & Path to your photo \\[2pt]
        \texttt{bio} & \textbf{One short sentence} about your research \\[2pt]
        \texttt{email} & Your BCM email \\[2pt]
        \texttt{linkedin} & Your profile URL \\[2pt]
        \texttt{github} & Your GitHub URL \\[2pt]
        \texttt{weight} & Lower = appears first \\[4pt]
      \end{tabularx}
      }

      \vspace{0.5em}
      \textbf{\color{secondary} Example file:}

      {\tiny \texttt{github.com/LiuzLab/Liuzlab.github.io/blob/main/\\ src/content/team/ap-hyun-hwan-jeong.md}}
    \end{column}
  \end{columns}
\end{frame}


\begin{frame}[fragile]{Step 3 (cont.): The Markdown Body}
  Below the \texttt{---} frontmatter, write your extended bio in \textbf{Markdown}:

  \begin{lstlisting}[style=yamlstyle, basicstyle=\ttfamily\tiny]
---
... (frontmatter above) ...
---
Dr. Jeong is an Assistant Professor specializing in
computational biology and bioinformatics.

His notable contributions include **SalmonTE**, an ultra-fast
pipeline for quantifying transposable elements from RNA-seq
data, and **CRISPRcloud**, a cloud-based platform for CRISPR
pooled screen analysis.
  \end{lstlisting}

  \vspace{0.5em}
  \begin{block}{Tips for the body text}
    \begin{itemize}
      \item Keep it to \textbf{2--3 short paragraphs}
      \item Use \texttt{**bold**} for key tools or project names
      \item This appears on your \textbf{full profile page}, not the team card
    \end{itemize}
  \end{block}
\end{frame}


\begin{frame}{Profile Fields Reference}
  \begin{table}
    \centering
    \scriptsize
    \begin{tabularx}{\textwidth}{l l X}
      \toprule
      \textbf{Field} & \textbf{Required?} & \textbf{Description} \\
      \midrule
      \texttt{name}           & Yes & Your full name \\
      \texttt{role}           & Yes & One of: \textit{PhD Student}, \textit{Postdoc}, \textit{Master Student}, etc. \\
      \texttt{avatar}         & Yes & Path to your photo in \texttt{src/assets/} \\
      \texttt{title}          & No  & Academic titles (array) \\
      \texttt{bio}            & No  & One-line description for the card \\
      \texttt{email}          & No  & Your email address \\
      \texttt{website}        & No  & Personal website URL \\
      \texttt{linkedin}       & No  & LinkedIn profile URL \\
      \texttt{github}         & No  & GitHub profile URL \\
      \texttt{twitter}        & No  & Twitter/X URL \\
      \texttt{googleScholar}  & No  & Google Scholar profile URL \\
      \texttt{weight}         & No  & Display order (lower = first, default 100) \\
      \bottomrule
    \end{tabularx}
  \end{table}

  \vspace{0.5em}
  {\footnotesize Valid \texttt{role} values: \textit{Principal Investigator, Professor, Associate Professor, Assistant Professor, Postdoc, Research Assistant, PhD Student, Master Student, Undergraduate, Alumni}}
\end{frame}


\begin{frame}[fragile]{Step 4: Add Your Photo}
  \begin{enumerate}
    \item Prepare a \textbf{square (1:1)} photo
    \item Name it descriptively: \texttt{jane-smith.jpg}
    \item Place it in the \texttt{src/assets/} directory
  \end{enumerate}

  \begin{lstlisting}[style=terminalstyle]
# Copy your photo into the assets folder
cp ~/Photos/my-headshot.jpg src/assets/jane-smith.jpg
  \end{lstlisting}

  \vspace{0.5em}
  Then reference it in your profile:

  \begin{lstlisting}[style=yamlstyle]
avatar: "../../assets/jane-smith.jpg"
  \end{lstlisting}

  \begin{block}{Photo tips}
    Professional headshot preferred. Supported formats: \texttt{.jpg}, \texttt{.png}, \texttt{.webp}
  \end{block}
\end{frame}


%==============================================================================
\section{How: Commit, Push, and Create a PR}

\begin{frame}[fragile]{Step 5: Stage and Commit Your Changes}
  \begin{lstlisting}[style=terminalstyle]
# Check what files you changed
git status

# Stage your profile and photo
git add src/content/team/phd-jane-smith.md
git add src/assets/jane-smith.jpg

# Commit with a descriptive message
git commit -m "Add Jane Smith member profile"
  \end{lstlisting}

  \vspace{0.5em}
  \begin{alertblock}{Commit message tips}
    \begin{itemize}
      \item Use imperative mood: ``Add'', ``Update'', ``Fix'' (not ``Added'')
      \item Be specific: ``Update Jane's bio and photo'' not ``Update stuff''
    \end{itemize}
  \end{alertblock}
\end{frame}


\begin{frame}[fragile]{Step 6: Push to Your Fork}
  \begin{lstlisting}[style=terminalstyle]
# Push your branch to your fork (origin)
git push origin update/your-name-info
  \end{lstlisting}

  \vspace{1em}
  This uploads your branch to \textbf{your fork} on GitHub --- it does \textbf{not} affect the main lab repository yet.
\end{frame}


\begin{frame}[fragile]{Step 7: Create a Pull Request}
  \begin{lstlisting}[style=terminalstyle]
# Create a PR from the command line
gh pr create \
  --title "Add Jane Smith member profile" \
  --body "Added my profile and headshot photo.

## Changes
- New file: src/content/team/phd-jane-smith.md
- New photo: src/assets/jane-smith.jpg

## Preview
Please preview the /team page after merging." \
  --repo LiuzLab/Liuzlab.github.io
  \end{lstlisting}

  \vspace{0.5em}
  The \texttt{gh pr create} command:
  \begin{itemize}
    \item Opens a PR from \textbf{your fork} $\rightarrow$ \textbf{upstream main}
    \item Adds a title and description
    \item Returns a \textbf{URL} you can share
  \end{itemize}
\end{frame}


\begin{frame}[fragile]{Step 8 (Optional): Test Locally}
  Want to preview the website before submitting a PR?

  \begin{lstlisting}[style=terminalstyle]
# Install dependencies (first time only)
npm install

# Start the local dev server
npm run dev
  \end{lstlisting}

  \vspace{0.5em}
  Visit \texttt{http://localhost:4321/team} in your browser to see your profile.

  \vspace{0.5em}
  \begin{block}{Requirements}
    Node.js 18+ and npm installed. Check with \lstinline[style=terminalstyle]{node --version}
  \end{block}
\end{frame}


\begin{frame}[fragile]{Checking PR Status and Review}
  \begin{lstlisting}[style=terminalstyle]
# View your PR in the browser
gh pr view --web

# Check the status of CI checks
gh pr checks

# List all open PRs on the repo
gh pr list --repo LiuzLab/Liuzlab.github.io
  \end{lstlisting}

  \vspace{0.5em}
  \textbf{What happens next:}
  \begin{enumerate}
    \item A reviewer (maintainer) looks at your changes
    \item They may request changes or approve
    \item Once approved, the PR is \textbf{merged} into \texttt{main}
    \item The website auto-deploys via GitHub Pages
  \end{enumerate}
\end{frame}


\begin{frame}{What Your PR Should Look Like}
  \begin{center}
    \includegraphics[width=0.95\textwidth]{figures/pr-example.png}
  \end{center}

  \vspace{0.5em}
  Your PR should include a clear title, description, and show the files you changed.
\end{frame}


\begin{frame}{Expected Output: Team Page}
  \begin{center}
    \includegraphics[width=0.9\textwidth]{figures/team-page.png}
  \end{center}

  \vspace{0.5em}
  Your profile will appear on the lab website team page like this!
\end{frame}


%==============================================================================
\section{How: The Full Workflow at a Glance}

\begin{frame}[fragile]{Quick Reference: Complete Workflow}
  \begin{lstlisting}[style=terminalstyle]
# 1. Fork & clone (one-time setup)
gh repo fork LiuzLab/Liuzlab.github.io --clone
cd Liuzlab.github.io

# 2. Sync and branch
git checkout main && git pull upstream main
git checkout -b update/your-name-info

# 3. Make your changes
cp ~/my-photo.jpg src/assets/your-name.jpg
# Edit src/content/team/role-your-name.md

# 4. Stage, commit, push
git add src/content/team/role-your-name.md
git add src/assets/your-name.jpg
git commit -m "Add your-name member profile"
git push origin update/your-name-info

# 5. Open a pull request
gh pr create --title "Add your-name profile" \
  --body "Added my profile." \
  --repo LiuzLab/Liuzlab.github.io
  \end{lstlisting}
\end{frame}


%==============================================================================
\section{Summary}

\begin{frame}{Summary}
  \begin{enumerate}
    \item \textbf{How:} Fork $\rightarrow$ Branch $\rightarrow$ Edit $\rightarrow$ Commit $\rightarrow$ Push $\rightarrow$ PR
    \item \textbf{Why:}
      \begin{itemize}
        \item[\color{primary}1.] \textbf{Decentralize} --- You own your profile. No bottleneck.
        \item[\color{primary}2.] \textbf{Learn GitHub} --- Same workflow used everywhere in software and research.
      \end{itemize}
  \end{enumerate}

  \vspace{1em}
  \begin{block}{Your action item}
    Update your profile on \texttt{LiuzLab/Liuzlab.github.io} and submit a PR!
  \end{block}
\end{frame}


%==============================================================================
\section{Troubleshooting}

\begin{frame}{Stuck? Ask AI for Help!}
  \textbf{\color{primary} If you encounter any Git/GitHub issues:}

  \vspace{0.5em}
  \begin{enumerate}
    \item \textbf{Ask an AI assistant:}
      \begin{itemize}
        \item Claude, GitHub Copilot, or Gemini
        \item Describe your error message or problem
        \item Get step-by-step solutions
      \end{itemize}
    
    \vspace{0.5em}
    \item \textbf{Common issues they can help with:}
      \begin{itemize}
        \item Merge conflicts --- ``How do I resolve this conflict?''
        \item Push rejected --- ``What does 'non-fast-forward' mean?''
        \item Commit mistakes --- ``How do I undo my last commit?''
        \item Branch issues --- ``How do I switch branches?''
        \item Workflow guidance --- ``Put all changes to a new branch, then make PR''
      \end{itemize}
  \end{enumerate}

  \vspace{0.5em}
  \begin{alertblock}{Pro Tip}
    Copy-paste error messages to AI. It will explain what happened and suggest fixes (force push, rebase, reset, etc.)
  \end{alertblock}
\end{frame}


\begin{frame}
  \centering
  \vspace{2cm}
  {\Huge \color{primary} Questions?}

  \vspace{1.5cm}
  {\large \texttt{github.com/LiuzLab/Liuzlab.github.io}}

  \vspace{1cm}
  {\large \href{https://liuzlab.github.io}{\textcolor{primary}{liuzlab.github.io}}}
\end{frame}

\end{document}
